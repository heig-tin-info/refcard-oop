\documentclass{article}

\usepackage[a4paper, landscape, margin=1cm]{geometry}
\usepackage{fontspec}
\usepackage[french]{babel}
\usepackage[fontsize=6.5pt]{scrextend}
\usepackage[T1]{fontenc}
\usepackage{multicol}
\usepackage{tabularx}
\usepackage{sectsty}
\usepackage{lmodern}
\usepackage{stix}
\usepackage{listings}
\usepackage{xcolor}
\usepackage{multirow}
\usepackage{titlesec}
\usepackage{fullpage}
\usepackage{tikz-uml}

\tikzumlset{
    fill class=white,
    fill template=white,
    fill package=white,
    fill note=white
}


\sloppy
\hyphenpenalty 10000000

\input{revision}

\setlength{\parskip}{0.2em}
\setlength{\parindent}{0em}

% Highlight configuration for C programming language
\lstset{
  language=python,
  breaklines=true,
  keywordstyle=\bfseries\color{black},
  basicstyle=\ttfamily\color{black},
  emphstyle={\em \color{gray}},
  emph={expr, type, NAME, ptr, name, expr, value, filename, label, member, type},
  mathescape=true,
  keepspaces=true,
  showspaces=false,
  showtabs=true,
  tabsize=3,
  columns=fullflexible,
  escapeinside={(*}{*)}
}

% Configuration
\renewcommand{\familydefault}{\sfdefault}

\sectionfont{\fontsize{12}{15}\selectfont}
\subsectionfont{\fontsize{10}{12}\selectfont}

\allsectionsfont{\sffamily\underline}

% No pages numbering
\pagenumbering{gobble}

% Titles and paragraphs more compact
\titlespacing*{\section}{0pt}{0pt}{0pt}
\titlespacing*{\subsection}{0pt}{0pt}{0pt}

\newlength\mybaselinestretch
\mybaselinestretch=0pt plus 0.02pt\relax
\addtolength{\baselineskip}{\mybaselinestretch}

\setlength\parindent{0pt}
\setlength\tabcolsep{1.5pt}
\setlength{\columnseprule}{0.4pt}

% Macros
\newcommand{\tab}{\hspace{2em}}
\newcommand{\etc}{\small \ldots}
\newcommand{\any}{$\hzigzag$~}
\newcommand{\spc}{$\mathvisiblespace$}
\newcommand{\cd}{\lstinline}

\begin{document}

\begin{multicols*}{3}

\begin{center}
  {\Large \bf Carte de référence Programmation Orientée Objets} \\
  HEIG-VD -- version \revision \ -- \revisiondate \\
\end{center}

Cette carte de référence peut être utilisée durant les travaux écrits
des cours utilisant \emph{python} à moins que le contraire soit explicitement formulé.
Elle est une liste non exhaustive du paradigme Objet.

Signification des termes utilisés dans cette carte de référence.

\begin{tabularx}{\linewidth}{
  >{\hsize=0.5\hsize}X% 10% of 4\hsize
  >{\hsize=1.5\hsize}X% 30% of 4\hsize
  >{\hsize=0.5\hsize}X% 30% of 4\hsize
  >{\hsize=1.5\hsize}X% 30% of 4\hsize
     % sum=4.0\hsize for 4 columns
  }

  \tt \etc      & Continuation logique    & \tt \any    & N'importe quoi d'accepté \\
  \tt /\any/    & Expression régulière    & \tt \spc    & Espace obligatoire \\
  \cd{type}     & \tt int, long, float, ... & \cd{name} & \tt /[A-Za-z][A-Za-z0-9\_]+/ \\
  \cd{value}    & Valeur & \cd{NAME} & \tt /[A-Z][A-Z0-9\_]+/ \\
  \cd{filename} & Chemin de fichier relatif & \cd{expr}   & e.g. \tt a + b \\
\end{tabularx}
\hrule

\section*{Class Diagram}

\begin{center}
\resizebox{\columnwidth}{!}{%
\begin{tikzpicture}
\begin{umlpackage}{p}
\begin{umlpackage}{sp1}
\umlclass[template=T]{A}{
  n : uint \\ t : float
}{}
\umlclass[y=-3]{B}{
  d : double
}{
  \umlvirt{setB(b : B) : void} \\ getB() : B}
\end{umlpackage}
\begin{umlpackage}[x=10,y=-6]{sp2}
\umlinterface{C}{
  n : uint \\ s : string
}{}
\end{umlpackage}
\umlclass[x=2,y=-10]{D}{
  n : uint
  }{}
\end{umlpackage}

\umlassoc[geometry=-|-, arg1=tata, mult1=*, pos1=0.3, arg2=toto, mult2=1, pos2=2.9, align2=left]{C}{B}
\umlunicompo[geometry=-|, arg=titi, mult=*, pos=1.7, stereo=vector]{D}{C}
\umlimport[geometry=|-, anchors=90 and 50, name=import]{sp2}{sp1}
\umlaggreg[arg=tutu, mult=1, pos=0.8, angle1=30, angle2=60, loopsize=2cm]{D}{D}
\umlinherit[geometry=-|]{D}{B}
\umlnote[x=2.5,y=-6, width=3cm]{B}{Je suis une note qui concerne la classe B}
\umlnote[x=7.5,y=-2]{import-2}{Je suis une note qui concerne la relation d'import}
\end{tikzpicture}
}
\end{center}

\end{multicols*}
\end{document}
