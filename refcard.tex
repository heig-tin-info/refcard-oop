\documentclass{article}

\usepackage[a4paper, landscape, margin=1cm]{geometry}
\usepackage{fontspec}
\usepackage[french]{babel}
\usepackage[fontsize=6.5pt]{scrextend}
\usepackage[T1]{fontenc}
\usepackage{multicol}
\usepackage{tabularx,booktabs}
\usepackage{sectsty}
\usepackage{lmodern}
\usepackage{stix}
\usepackage{listings}
\usepackage[dvipsnames]{xcolor}
\usepackage{multirow}
\usepackage{titlesec}
\usepackage{fullpage}
\usepackage{tikz-uml}
\usepackage{makecell}
\setcellgapes{3pt}
\let\oldmakecell\makecell
\renewcommand\makecell[2][cc]{\renewcommand\cellalign{#1}\oldmakecell{#2}}
\tikzumlset{
    fill class=white,
    fill template=white,
    fill package=white,
    fill note=white
}

\sloppy
\hyphenpenalty 10000000

\ifdefined\tinted
  \pagecolor{Lavender!60}
\fi

\input{revision}

\setlength{\parskip}{0.2em}
\setlength{\parindent}{0em}

% Highlight configuration for C programming language
\lstset{
  language=python,
  breaklines=true,
  keywordstyle=\bfseries\color{black},
  basicstyle=\ttfamily\color{black},
  emphstyle={\em \color{gray}},
  emph={expr, type, NAME, ptr, name, expr, value, filename, label, member, type},
  mathescape=true,
  keepspaces=true,
  showspaces=false,
  showtabs=true,
  tabsize=3,
  columns=fullflexible,
  escapeinside={(*}{*)}
}

% Configuration
\renewcommand{\familydefault}{\sfdefault}

\sectionfont{\fontsize{12}{15}\selectfont}
\subsectionfont{\fontsize{10}{12}\selectfont}

\allsectionsfont{\sffamily\underline}

% No pages numbering
\pagenumbering{gobble}

% Titles and paragraphs more compact
\titlespacing*{\section}{0pt}{0pt}{0pt}
\titlespacing*{\subsection}{0pt}{0pt}{0pt}

\newlength\mybaselinestretch
\mybaselinestretch=0pt plus 0.02pt\relax
\addtolength{\baselineskip}{\mybaselinestretch}

\setlength\parindent{0pt}
\setlength\tabcolsep{1.5pt}
\setlength{\columnseprule}{0.4pt}

% Macros
\newcommand{\tab}{\hspace{2em}}
\newcommand{\etc}{\small \ldots}
\newcommand{\any}{$\hzigzag$~}
\newcommand{\spc}{$\mathvisiblespace$}
\newcommand{\cd}{\lstinline}

\begin{document}

\begin{multicols*}{3}

  \begin{tabularx}{\columnwidth}{@{}lX}
    \raisebox{-\totalheight}{\includegraphics[width=1cm]{assets/heig-vd-black.pdf}} &
\begin{center}
  {\Large \bf Carte de référence Programmation Objet} \\
  version \revision \ -- \revisiondate \\
\end{center}
\end{tabularx}
{
\scriptsize
Cette carte de référence est une liste non exhaustive du paradigme Objet et de ses notions fondamentales
}

\hrule

\section*{S.O.L.I.D.}
\begin{description}%
\item[S] \emph{Responsabilité unique}. Le principe de responsabilité unique stipule qu'une classe ne devrait avoir qu'\textbf{une seule raison de changer} et \textbf{une seule responsabilité}.
\item[O] \emph{Ouvert/Fermé}. Une entité applicative (classe, fonction, module ...) doit être fermée à la modification directe mais ouverte à l'extension
\item[L] \emph{Substitution de Liskov}. Une instance de type T doit pouvoir être remplacée par une instance de type G, tel que G sous-type de T, sans que cela ne modifie la cohérence du programme
\item[I] \emph{Ségrégation des interfaces}. Préférer plusieurs interfaces spécifiques pour chaque client plutôt qu'une seule interface générale
\item[D] \emph{Inversion des dépendances}. Il faut dépendre des abstractions, pas des implémentations.
\end{description}

\section*{Concepts objets}
\begin{description}
\item[Objets] Les \emph{Objets} sont des instances de classes résidentes en mémoire. Ils possèdent un constructeur et un destructeur.
\item[Class] Les classes sont des \emph{Modèles} d'\emph{Objets}. Ils sont définis par des attributs et des méthodes. 
\item[Héritage] Les classes peuvent hériter d'une autre classe et ainsi se spécialiser.
\item[Polymorphisme] Une classe parente peut être substituée par une classe enfant et ainsi le comportement de l'objet est généralisé.
\item[Abstraction] Offre une abstraction des détails d'implémentation par la description d'interfaces et de comportements. 
\item[Encapsulation] Permet de regrouper les données avec des routines qui en permettent la manipulation. Elle permet de modifier les structures de données internes sans modifier l'interface de celle-ci. 
\end{description}
\section*{Relations}
\begin{tabularx}{\columnwidth}{lX}
\begin{tikzpicture}
\umlemptyclass{A}
\umlemptyclass[x=3]{B}
\umldep{A}{B}
\end{tikzpicture} & Dépendence \\
\begin{tikzpicture}
\umlemptyclass{A}
\umlemptyclass[x=3]{B}
\umlassoc{A}{B}
\end{tikzpicture} & Association \\ 
\begin{tikzpicture}
\umlemptyclass{A}
\umlemptyclass[x=3]{B}
\umluniassoc{A}{B}
\end{tikzpicture} & Association Unique \\ 
\begin{tikzpicture}
\umlemptyclass{A}
\umlemptyclass[x=3]{B}
\umlaggreg{A}{B}
\end{tikzpicture} & Aggrégation. La classe A contient une collection de B. Lorsque A est détruite les instances de B ne sont pas détruites. \\ 
\begin{tikzpicture}
\umlemptyclass{A}
\umlemptyclass[x=3]{B}
\umlcompo{A}{B}
\end{tikzpicture} & Composition. La classe A contient une collection de B. Lorsque A est détruite les instances de B sont également détruites. \\
\begin{tikzpicture}
\umlemptyclass{A}
\umlemptyclass[x=3]{B}
\umlinherit{A}{B}
\end{tikzpicture} & Héritage. La classe A hérite de B. On dit que A et un B.\\
\begin{tikzpicture}
\umlemptyclass{A}
\umlemptyclass[x=3]{B}
\umlimpl{A}{B}
\end{tikzpicture} & Implémentation/Réalisation \\
\end{tabularx}


\section*{Modèle de classes}
\begin{center}
\resizebox{\columnwidth}{!}{%
\begin{tikzpicture}
\begin{umlpackage}{p}
\begin{umlpackage}{sp1}
\umlclass[template=T]{A}{
  n : uint \\ t : float
}{}
\umlclass[y=-3]{B}{
  d : double
}{
  \umlvirt{setB(b : B) : void} \\ getB() : B}
\end{umlpackage}
\begin{umlpackage}[x=10,y=-6]{sp2}
\umlinterface{C}{
  n : uint \\ s : string
}{}
\end{umlpackage}
\umlclass[x=2,y=-10]{D}{
  n : uint
  }{}
\end{umlpackage}

\umlassoc[geometry=-|-, arg1=tata, mult1=*, pos1=0.3, arg2=toto, mult2=1, pos2=2.9, align2=left]{C}{B}
\umlunicompo[geometry=-|, arg=titi, mult=*, pos=1.7, stereo=vector]{D}{C}
\umlimport[geometry=|-, anchors=90 and 50, name=import]{sp2}{sp1}
\umlaggreg[arg=tutu, mult=1, pos=0.8, angle1=30, angle2=60, loopsize=2cm]{D}{D}
\umlinherit[geometry=-|]{D}{B}
\umlnote[x=2.5,y=-6, width=3cm]{B}{Je suis une note qui concerne la classe B}
\umlnote[x=7.5,y=-2]{import-2}{Je suis une note qui concerne la relation d'import}
\end{tikzpicture}
}
\end{center}

\end{multicols*}
\end{document}
